\section*{Introducción}

Una razón por la que Octave es un lenguaje muy potente es que se caracteriza por ser un programa funcional. Esto quiere decir que tiene múltiples funciones que pueden ser aplicadas a objetos más complejos que un número, por ejemplo una matriz. Una matriz es una herramienta matemáticas que permite tratar mucha información de una manera muy eficiente y flexible. Por ejemplo, una matriz de pixeles puede ser una imagen, o una película; o una matriz de fluctuaciones puede ser un sonido o una voz humana. De este modo vamos a ver en este número como se tratan las matrices en Octave y algunas funciones asociadas a ellas. 

\section{Matrices}

Una matriz de números es un objeto matemático que contiene números colocados por filas y columnas. En octave estas matrices irán entre corchetes($[\;]$ y dentro escribiremos las filas separados por espacio o comas y las columnas por punto y coma. Veamos algunos ejemplos,
\begin{octavebox}
  \begin{verbatim}
octave:1> A=[1 2 3;4 5 6]
A =

   1   2   3
   4   5   6

\end{verbatim}
\end{octavebox}

\begin{octavebox}
  \begin{verbatim}
octave:2> b=[1 10 100]
b =

     1    10   100

\end{verbatim}
\end{octavebox}

De este modo hemos creado dos matrices: $A$ con $2$ filas y $3$ columnas, y la matriz $b$ con una fila y $3$ columnas (esta matriz más propiamente es llamada {\emph vector fila}). Basado en la posición en la que se encuentran los elementos de las matrices podemos extraer sus elementos o bien trozos de la matriz, lo que se denominan submatrices.

\begin{octavebox}
  \begin{verbatim}
octave:3> A(2,3)
ans =  6
octave:4> A(:,[1 3])
ans =

   1   3
   4   6
\end{verbatim}
\end{octavebox}

Cuando escribimos dos puntos(esto es,  $:$) significa que tomamos o bien todas las filas o bien todas las columnas según corresponda.
\subsection{Operaciones con matrices}
Pues las matrices si son del mismo tamaño se pueden sumar, restar o multiplicar por un escalar.

\begin{octavebox}
\begin{verbatim}
octave:5> A+A
ans =

    2    4    6
    8   10   12

octave:6> 2*A
ans =

    2    4    6
    8   10   12
\end{verbatim}
\end{octavebox}

Lo que no podemos es multiplicar dos matrices cualesquiera, sino que el número de columnas de la de la izquierda debe coincidir con el número de filas de la de la derecha. ¿Recuerdas eso? 

\begin{octavebox}
\begin{verbatim}
octave:7> A*b
error: operator *: nonconformant arguments (op1 is 2x3, op2 is 1x3)
octave:7> A*b'
ans =

   321
   654

octave:8> A'*A
ans =

   17   22   27
   22   29   36
   27   36   45

\end{verbatim}
\end{octavebox}

Observa que la prima $'$ lo que hace es transponer y conjugar la matriz, esto es cambiar las filas por columnas y viceversa y si los números de las matrices fuesen complejos los pasaría a sus conjugados.

\newpage 
Como hemos dicho antes \emph{Octave} es funcional. Fíjate que de una vez podemos calcular el cuadrado, o el cubo, o el seno trigonométrico, o la tangente, o la raíz cuadrada de todos los elementos de la matriz $A$, por ejemplo:

\begin{octavebox}
\begin{verbatim}
octave:9> A.^2
ans =

    1    4    9
   16   25   36

octave:10> A.^3
ans =

     1     8    27
    64   125   216

octave:11> sin(A)
ans =

   0.84147   0.90930   0.14112
  -0.75680  -0.95892  -0.27942

octave:12> tan(A)
ans =

   1.55741  -2.18504  -0.14255
   1.15782  -3.38052  -0.29101

octave:13> sqrt(A)
ans =

   1.0000   1.4142   1.7321
   2.0000   2.2361   2.4495.

\end{verbatim}
\end{octavebox}

\newpage
La última operación usual en matrices es la inversa de una matriz. No todas las matrices tienen inversa; estas tienen que ser cuadradas (mismo número de filas que de columnas) y además su determinante debe ser distinto de cero. En este caso podemos calcular la inversa y se calcula con el comando $inv$.

\begin{octavebox}
\begin{verbatim}
octave:15> C=[1 2;2 5]
C =

   1   2
   2   5

octave:16> det(A)
error: det: argument must be a square matrix
octave:16> C=[1 2;2 5]
C =

   1   2
   2   5

octave:17> det(C)
ans =  1
octave:18> inv(C)
ans =

   5  -2
  -2   1

octave:19> ans*C
ans =

   1   0
   0   1
\end{verbatim}
\end{octavebox}

\subsection{Generando algunas matrices}
Tenemos muchos tipos de matrices:
\begin{itemize}
\item Caracterizadas por su forma: cuadradas, rectangulares, triangulares superiores,  triangulares inferiores, escalonadas, etc.
\item Caracterizadas por su contenido: diagonales, tridiagonales, huecas o sparse, aleatorias,etc.
\item Caracterizadas por propiedades intrìnsecas: Pascal, Vandermonde, Hilbert, Toeplitzs, Hankel, etc.  
\end{itemize}
\begin{wrapfigure}{r}{0.5\textwidth} 
\begin{mybox}
  \begin{tabular}{cc}
    \begin{minipage}{0.3\linewidth}
      {\fontsize{100}{60}\selectfont\color{redsol} \ManFace}
      {\Huge }
    \end{minipage}&
    \begin{minipage}{0.5\linewidth}
      ¿Quieres ir al {\color{redsol} supermercado} a comprar gratis
      algunas matrices?
    \end{minipage}
  \end{tabular}\\
  \centering 
  Esta es la dirección:\\ 
  \url{math.nist.gov/MatrixMarket}
%\includegraphics[scale=0.4]{topimage.jpg}\\
    
\end{mybox}
\end{wrapfigure}
Os animamos a que investiguéis los siguientes instrucciones con nuevos comandos en octave: $rand(2,3)$, $diag([1\; 2\; 3\; 4])$, $tril(A)$, $triu(A)$, $ones(2,3)$, $zeros(2,3)$, $eye(3)$, $vander([1\; 2\; 3\; 4])$ y $pascal(3)$.
ALguna pista de lo que va a salir puedes mirarlo en lo que sigue.
\begin{octavebox}
\begin{verbatim}
octave:21> A1=rand(2,3);A2=diag([100 200]);A3=tril(A);
A4=triu(A);A5=ones(2,3);A6=zeros(2,3);A7=eye(2);
octave:22> %¿Las ponemos todas en una sola?
octave:22> [A1 A2 A3 A4 A5 A6 A7]
ans =

 Columns 1 through 7:

     0.66373     0.80437     0.18190   100.00000     0.00000     1.00000     0.00000     
     0.08936     0.00524     0.91602     0.00000   200.00000     4.00000     5.00000     

Columns 8 through 14:

     0.00000     1.00000      2.00000     3.00000     1.00000     1.00000     1.00000
     0.00000     0.00000      5.00000     6.00000     1.00000     1.00000     1.00000

 Columns 15 through 19:

     0.00000     0.00000     0.00000     1.00000      0.00000
     0.00000     0.00000     0.00000     0.00000      1.00000

\end{verbatim}
\end{octavebox}


\subsection{Algunos comandos interesantes}
La aplicabilidad de las matrices es enorme y en la ingeniería una matriz aparece en múltiples aplicaciones, desde la resolución de sistemas de ecuaciones, la resolución de ecuaciones diferenciales, el procesamiento de señales, simulaciones, etc. En futuros números estudiaremos algunos problemas en ingeniería donde podamos ver la importancia de las matrices y el uso de \emph{Octave} en ella. Por el momento, os animamos a que vayáis investigando algunos comandos como son \emph{eig} que calcula los autovalores de una matriz, \emph{svd} que calcula la descomposiciòn en valores singulares o \emph{lu} que calcula la factorización de una matriz basada en la eliminaciòn gaussiana.

\section{Procesando una foto}

¿Qué es una foto? Es una colección ordenada de cuadraditos del mismo tamaño, cada uno de ellos coloreado de un color determinado. Cuanto más pequeño es el cuadrado mayor resolución de la foto y cuanto más grande, menor resolución y la calidad de la foto no será tan buena. Entonces podríamos pensar en una foto como un rectángulo lleno de cuadraditos del mismo tamaño y que cada cuadrado tiene la información del color del cuadrado, ¿verdad? Es decir una foto es una matriz de colores. Vamos a tomar una foto, ver esa matriz que está escondida en ella y jugaremos un poco con la imagen. Para ello en vez de trabajar como hasta ahora, que hemos ido escribiendo en la pantalla de ejecución de octave cada instrucción por separado, ahora vamos a juntar muchas instrucciones juntas en un archivo, que se llama \emph{script}, que tiene extensión \emph{.m} y llamaremos en la pantalla de ejecución de octave a este archivo por su nombre. En windows te recomendamos que utilices como editor \emph{notepad++} (¡Cuidado no se pueden usar editores estilo word ya que estos guardan formatos además del texto!); y en linux, por ejemplo os recomendamos utilizar el editor emacs. EL programa que hemos generado es el siguiente:

\begin{octavebox}
\begin{verbatim}
% El % lo usamos para hacer comentarios, esto no afecta a octave, no
% lee ni interpretalo que haya en las lìneas que comienzan con %

clear all
% Vamos a crear la matriz A que representa la foto "fotoNena.jpg"
A=imread("fotoNena.jpg");

%El ; lo colocamos para que no saque por pantalla la matriz A

%Veamos cuantas filas y columnas tiene la matriz A
disp("La foto està presentada en la matriz A de tamaño:")
size(A)
% Vamos a crear una figura con una fila y tres columnas
subplot(1,3,1)
% en la primera ponemos la foto "fotoNena.jpg"
imshow(A)

% en la segunda vamos a mostrarla rotada 45º

b=imrotate(A,45,'bilinear','crop');
subplot(1,3,2)
imshow(b)

% la pasamos a blanco y negro

c=rgb2gray(A);
subplot(1,3,3)
imshow(c) 
\end{verbatim}
\end{octavebox}

\newpage
Ahora mostramos en la Figura 1 la salida que genera el programa anterior.

\begin{figure}[ht!]
  \centering
  \begin{figurebox}
  \centering
  \includegraphics[width=\textwidth]{output.png}
  \caption{Pantallazo de la salida al ejecutar el programa}
  \label{fig:2}
  \end{figurebox}
\end{figure}
  

%\vspace{3cm}
%\noindent
%\includegraphics[width=\textwidth]{pubmm2.png}

\newpage
%%% Local Variables: 
%%% mode: latex
%%% TeX-master: "informaticaeningenieria"
%%% End: 



