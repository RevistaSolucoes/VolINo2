\section{Introducción}

\begin{wrapfigure}{r}{0.5\textwidth} 
\vspace{-1cm}
  \begin{figurebox}
    \vspace{0.1cm}
  \centering
  \includegraphics[width=\textwidth]{TuberiaMolecor.jpg}
  \caption{Tuberías  TOM\textcopyright \,  PVC-0 de Molecor.}
  \label{fig:1}
\end{figurebox}
\end{wrapfigure}En esta ocasión os presentamos una nueva generación de tuberías, estás son las tuberías TOM\textcopyright \,  PVC-0 de Molecor (veáse Fígura 1). Estas tubeías son utilizadas para el transporte del agua, principalmente en abastecimiento, irrigación y saneamiento.  Hasta el momento de la aparición de estas, los procesos de fabricación de canalizaciones de PVC Orientado (PVC-O) presentaban dificultades y en algunos casos imposibilidad para fabricar tuberías de grandes diámetros y alta presión. La mejora tecnológica que presenta el nuevo sistema de fabricación permite una mejora de la calidad de la tubería fabricada. Además, las tuberías TOM\textcopyright fabricadas por Molecor conservan totalmente inalterada la calidad del agua que circula por su interior. Esto se debe a que su material es químicamente inerte y resistente a la corrosión, por lo que no se producen  degradaciones ni migraciones del  material hacia el agua que transportan.

Molecor (\url{www.molecor.com}) es una empresa veterana con tecnología al servicio del agua y la cuál mantiene acuerdos con los más reputados centros públicos de investigación y desarrollo en España, como el Centro para el Desarrollo Tecnológico e Industrial y la Empresa Nacional de Innovación. La aparición de esta nueva tecnología que os presentamos está patentada nivel mundial por Molecor. 


\section{Propiedades}
Podemos destacar las siguientes especificaciones de esta nueva generación de tuberías:
\begin{itemize}
\item Material: policloruro de vinilo
\item  El límite inferior de Confianza de la Resistencia Hidrostática prevista ($\sigma_{LPL}$) es el valor con la unidad de esfuerzo, el cual representa el 97,5\% del límite inferior de confianza de la resistencia hidrostática prevista para un valor de temperatura $T$ y durante un tiempo $t$.
\item Máxima flexibilidad: permite soportar deformaciones de hasta el 100\% del diámetro interior. Veáse Figura 2.
\item El PVC-O es un material significativamente más fuerte y resistente que el PVC-U y por tanto proporciona mejor comportamiento final. Es casi el \emph{doble de resistente a tensión} y \emph{casi diez veces más resistente al impacto que el PVC-U}.
\item El PVC-O se fabricó por primera vez a finales de 1970. Desde entonces se han llevado a cabo numerosos desarrollos que han mejorado la productividad y aumentando por tanto la rentabilidad. 

\begin{wrapfigure}{r}{\textwidth} 
  \begin{mybox}
    $\sigma_{LPL}$ es un parámetro/propiedad que se emplea para la
    fabricación de tuberías, que más o menos viene a decir que si esa tubería
    lleva agua a 20º, te asegura que a los 50 años de uso, su resistencia
    hidrostática como mínimo estará al 97,5\% del valor marcado inicialmente
\end{mybox}
\end{wrapfigure}
\item Esta nueva tecnología, le permite fabricar tubería de diámetros nominales de 500 y 630 mm, convirtiéndose así en el primer fabricante a nivel mundial de estas canalizaciones, ya que hasta ahora la gama disponible en el mercado sólo llegaba a diámetro 400mm. Las ventajas que ofrece este material, como su menor coste, tanto del producto como de su instalación, o su mayor rendimiento de instalación, lo convierten en la mejor opción para la realización de obras de este tipo. 
\end{itemize}
\begin{wrapfigure}{r}{0.3\textwidth}
\vspace{-1cm} 
  \begin{figurebox}
  \centering
  \includegraphics[scale=0.5]{felxible.jpg}
  \caption{Flexibilidad.}
  \label{fig:1}
\end{figurebox}
\end{wrapfigure}
\section{Las ventajas del PVC TOM\textcopyright Orientado}
Las ventajas que ofrece el material se deben por una parte a su naturaleza química y por otra a la mejora de propiedades que se producen durante su fabricación gracias a la orientación molecular.

El proceso de orientación molecular, además de mejorar de forma importante las propiedades mecánicas del tubo, produce una disminución del espesor de la pared del mismo. De esta forma, las tuberías tienen un menor peso y esto hace que puedan ser manipuladas e instaladas más fácilmente y de forma manual hasta diámetro DN250mm. Para diámetros mayores, aunque es necesario un elemento mecánico para facilitar este proceso, no es necesario disponer de una grúa de gran tonelaje como por ejemplo es necesario utilizar en el caso de la fundición dúctil.

Por otra parte, también hay que destacar, que el tubo de PVC-O es muy resistente al impacto por golpes y a la propagación de grietas, esto hace que se minimicen de forma significativa las roturas durante su manipulación e instalación en obra.

De igual forma, el eficaz diseño de la copa realizado por Molecor, hace que la junta de estanqueidad quede perfectamente instalada y que la conexión entre los tubos se realice de forma más rápida. La conexión de los tubos se realiza por enchufe tipo campana. Esta facilidad de conexión, hace que no sea necesaria la utilización de mano de obra especializada ni maquinaria específica para la instalación, como por ejemplo es el caso del polietileno en el que se necesita una máquina de soldadura y operarios especializados. Influyendo mucho también en ese caso, las condiciones climatológicas, en concreto la humedad. 

\begin{wrapfigure}{r}{0.4\textwidth} 
\vspace{-1cm}  \begin{figurebox}
  \centering
  \includegraphics[scale=0.6]{Comparacion.jpg}
  \caption{PVC-Orientado, PVC-Orientado TOM\textcopyright y PVU.}
  \label{fig:1}
\end{figurebox}
\end{wrapfigure}


El PVC es un material químicamente inerte frente a los productos presentes en la naturaleza, de forma que no se produce corrosión durante su larga vida útil, por lo que no es necesaria la utilización de recubrimientos protectores. Así, no hay que preocuparse especialmente por la calidad del suelo donde vayan a ir enterradas las tuberías, ni por la calidad del agua que circula por su interior, por lo que son perfectamente válidas para el transporte de agua tanto de consumo humano como agua residual. De esta forma, se asegura que nunca se van a producir puntos de corrosión que pueden alterar la calidad del agua.

Por otra parte, hay que tener en cuenta que la reducción del espesor de la pared del tubo, hace que aumente de forma considerable la capacidad hidráulica de la conducción  que varía entre el 15\% y el 40\% dependiendo del material y el diámetro que se compare. Por otra parte, también se da el hecho de que las pérdidas de carga producidas son mucho menores y por tanto podemos realizar el transporte a mayor velocidad, con lo que también se aumenta la capacidad de la red.

\begin{wrapfigure}{r}{0.5\textwidth} \vspace{-1cm}
  \begin{figurebox}
  \centering
  \includegraphics[width=\textwidth]{AcopioAcoplo.jpg}
  \caption{Acopio y Acoplo de las tuberías}
  \label{fig:1}
\end{figurebox}
\end{wrapfigure}

Una ventaja a destacar, sobre todo al comparar con la fundición dúctil, es el mejor comportamiento que tiene el PVC-O frente al golpe de ariete, que llega a ser hasta 3 veces inferiores, con lo que la seguridad de todos los elementos de la red aumenta de forma considerable en los cierres y aperturas de válvulas.

De igual manera, es perfectamente válido para transportar agua de consumo humano, en su gama de tubería azul, como para agua de riego  y agua reutilizada en su gama de tubería morado. 

Por otra parte la completa estanqueidad de las uniones, debido tanto a la gran calidad de la junta elástica utilizada, como al eficaz diseño de las copas, también contribuye a evitar fugas del agua canalizada.

De esta forma,  la vida útil de la tubería se mantiene intacta, pudiendo decirse que son la herramienta perfecta para la gestión de los recursos hídricos disponibles, con lo que se contribuye de forma importante a la sostenibilidad del planeta.

\section{Instalación}

Podemos ver la instalación de estas tuberías en tan sólo cinco minutos en este vídeo:\\ {\small\url{http://www.youtube.com/watch?v=p85ZXqo81tU&feature=c4-overview&list=UUPjpqyZ_56RKVBsc-_hWHVg}}.

\section{Las tuberías TOM\textcopyright \, PVC-0 de Molecor en Ángola}

SAEMA ya ha llevado este tipo de tuberías para diferentes instalaciones de tratamiento y abastecimiento de aguas: en las localidades de
Fazil, Tchimbambo y Santa Ana (Proyecto ETAPS Benguela-Sea), en la localidad Lombe (Proyecto ETAP Lombe-Sea), en provincia de Cabinda (Proyecto de rehabilitación y ampliación optimizada, ETA I y II). 

En estos momentos Saema está llevando a cabo proyectos basados en estas tecnologías:  el PROYECTO ETAPS UIGE-SEA para el tratamiento y abastecimiento de las localidades de Masseque, Sole y Casamanda.Kiwenbo-Kimbengui. 
%Se muestran algunas fotos en la Figura \ref{fig:1}
%\begin{center}
%\begin{wrapfigure}{r}{0.5\textwidth} 
%  \begin{figurebox}
%  \centering
%  \includegraphics[width=\textwidth]{angola1.jpg}   
%   \caption{Tuberías Tom\textcopyright PVC en Ángola}
%  \label{fig:1}
%\end{figurebox}
%\end{wrapfigure}
%\end{center}
%\begin{wrapfigure}{r}{0.5\textwidth} 
%  \begin{figurebox}
%  \centering
%  \includegraphics[width=\textwidth]{angola2.jpg}
%   \caption{Tuberías Tom\textcopyright PVC en Ángola}
%  \label{fig:1}
%\end{figurebox}
%\end{wrapfigure}
\section{Un par de conceptos técnicos}
\begin{itemize}
\item {\bf Resistencia a la tensión:} Es el resultado obtenido cuando a un material se
le somete a esfuerzos de tracción para ver cuál es su límite antes de
romperse y las deformaciones que sufre en función de la tensión aplicada
pero sin llegar a romperse. por ejemplo si curvamos un tramo de tubería de PVC-O y otra de
PVC normal, cuando la del normal se rompa, la de PVC-O aún podrá seguir
doblándose el doble. Y esto nos beneficia, en que nos podemos ahorrar
dinero, es decir, si tenemos que llevar una tubería por una zanja con una
cierta inclinación ascendente, si la tubería es muy rígida hay que instalar
tramos planos y unirlos con codos a 11º o así para ir absorbiendo la curva
porque si la forzáramos a doblarse se rompería, sin embargo, si permite esa
curvatura sin romperse ni perder propiedades nos ahorramos ese codo.
\item {\bf Resistencia al impacto:} Es la propiedad de las tuberías (y de cualquier
material) a absorber los golpes que puedan recibir sin romperse (en nuestro
caso, caída de piedras o herramienta, caída de la tubería al suelo, etc.).
Depende de muchos factores como la temperatura, la forma del material y por
lo tanto el punto dónde se golpee, etc., en este caso vendría a decir que
para una misma tubería, por ejemplo de diámetro 90 mm, hecha en PVC, la de
PVC-O aguanta valores 10 veces más altos que la convencional. Pero además es
que si la comparamos con otros materiales también obtiene mejores datos.
\end{itemize}

\vspace{3.75cm} 
\noindent
\includegraphics[width=\textwidth,scale=0.5]{pubmm.png}

%%% Local Variables: 
%%% mode: latex
%%% TeX-master: "novedades"
%%% End: 


